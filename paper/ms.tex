\documentclass{emulateapj}
\submitted{{\it Submitted for publication in ApJ}}
\usepackage{multirow,color,wrapfig,ulem}
\usepackage {graphicx}
\usepackage{graphics}
\usepackage{amsmath}
\usepackage[dvips]{epsfig}
\bibliographystyle{apj}
\newcommand{\avg}[1]{\langle{#1}\rangle}  
\newcommand{\nscatt}{\langle N_{\rm  scatt}\rangle}
\newcommand{\ly}{{\ifmmode{{\rm Ly}\alpha~}\else{Ly$\alpha$~}\fi}}
\newcommand{\hMpc}{{\ifmmode{h^{-1}{\rm Mpc}}\else{$h^{-1}$Mpc }\fi}}   
\newcommand{\hGpc}{{\ifmmode{h^{-1}{\rm Gpc}}\else{$h^{-1}$Gpc }\fi}}   
\newcommand{\hmpc}{{\ifmmode{h^{-1}{\rm Mpc}}\else{$h^{-1}$Mpc }\fi}}  
\newcommand{\hkpc}{{\ifmmode{h^{-1}{\rm kpc}}\else{$h^{-1}$kpc }\fi}}  
\newcommand{\hMsun}{{\ifmmode{h^{-1}{\rm
        {M_{\odot}}}}\else{$h^{-1}{\rm{M_{\odot}}}$}\fi}}   
\newcommand{\hmsun}{{\ifmmode{h^{-1}{\rm
        {M_{\odot}}}}\else{$h^{-1}{\rm{M_{\odot}}}$}\fi}}   
\newcommand{\Msun}{{\ifmmode{{\rm {M_{\odot}}}}\else{${\rm{M_{\odot}}}$}\fi}}  
\newcommand{\msun}{{\ifmmode{{\rm {M_{\odot}}}}\else{${\rm{M_{\odot}}}$}\fi}}  
\newcommand{\lya}{{Lyman $\alpha$~}}
\newcommand{\clara}{{\texttt{CLARA}}~}
\newcommand{\rand}{{\ifmmode{{\mathcal{R}}}\else{${\mathcal{R}}$ }\fi}}  
\newcommand{\hs}{{\hspace{1mm}}}  
\newcommand{\kms}{{\ifmmode{{\mathrm{\,km\ s}^{-1}}}\else{\,km~s$^{-1}$}\fi}}
% definition to produce a "less than or similar to" symbol
\def\lsim{~\rlap{$<$}{\lower 1.0ex\hbox{$\sim$}}}
% definition to produce a "greater than or similar to" symbol
\def\gsim{~\rlap{$>$}{\lower 1.0ex\hbox{$\sim$}}}
%@arxiver{fig3.pdf,fig11a.pdf, fig11b.pdf} 
\begin{document}



\title{The connection between star formation rate and dark matter halo
  mass in the epoch of reionization}  
\shorttitle{SFR and DM halo mass during reionization}

\shortauthors{Gomez-Cortes, Forero-Romero}

\author{Felipe L. Gomez-Cortes, Jaime E. Forero-Romero}  
\affil{Departamento de F\'{i}sica, Universidad de los Andes, Cra. 1
No. 18A-10, Edificio Ip, Bogot\'a, Colombia}
\email{fl.gomez10@uniandes.edu.co}
\email{je.forero@uniandes.edu.co}

\keywords{galaxies: high-redshift --- methods: numerical} 
\begin{abstract}
We present updated constraints on the relationship between the star
formation rate and dark matter halo mass at redshift $z\sim 6$.
The observational basis for our work is the restframe UV luminosity
function data obtained with CFHTLS, HST Legacy Survey and UKIDSS.
The constraints are based on an abundance matching methodology to the
observational data using cosmological N-body simulations.
We also take into account the influence on the results of the dust extinction
scaling derived from observations by Bouwens.
We compare our results against the results of abundance matching
methods (to the stellar mass), a semi-analytic model of galaxy
formation (GALFORM) and a hydrodynamical simulation (Illustris).
\end{abstract}


\section{Introduction}
\label{sec:intro}



All magnitudes are in AB system.
\section{Observational Datasets}
\label{sec:theo}


We use information from seven observational data sets.
Four obtained with the Hubble Space Telescope (HST) and three from
ground-based telescopes.
All of them select galaxy candidates at $z\sim 6$ using the drop-out
technique \citep{Steidel96}.  

The data from the Hubble Space Telescope Legacy (HSTL)
\citep{Bouwens15} is a compilation   of observations since the
installation of the Advanced Camera for Surveys (ACS) in 2002,
through the near-infrared Wide Field Camera 3 (WFC3/IR) installed in
2009, up to 2012.  

The HST fields of view are: XDF, HUDF09-1, HUDF09-2, CANDELS-S/Deep, 
CANDELS-S/Wide, ERS, CANDELS-N/Deep, CANDELS-N/Wide, CANDELS-UDS, 
CANDELS-COSMOS and CANDELS-EGS, with areas of 4.7, 4.7, 4.7, 64.5, 34.2, 
40.5, 62.9, 60.9, 151.2, 151.9 and 150.7 arcmin$^2$ respectively. The total area 
at redshift 6 corresponds 
to 740.8 arcmin$^2$ over five different lines of sight, wit a total volume of 
$1.8 \times 10^6 \textrm{Mpc}^3$.
Two cameras performed the observations: ACS and WFC3/IR, using 
$B_{435}$, $V_{606}$, $ i_{814}$, $ z_{850}$, $ I_{814}$, $Y_{098}$, $Y_{105}$, $J{125}$, 
$JH_{140}$ and $H_{160}$ filters.
The limit magnitude is between $\sim27.5$mag in CANDELS-EGS and$\sim30 $mag in the 
deepest field (XDF). Total number of z=6 LBG candidates is 940, most of them in the faint 
end of the LF, magnitudes in the rest frame are in the range $-22.52\leq M_{1600} \leq -16.77$
\citet{Bouwens15} calculated LF using a stepwise maximum-likelihood (SWML) based on \citet{Efstathiou88}. 
The Schechter parameters derived are: 
$\phi^* =(0.33_{-0.10}^{+0.15}) \times 10 ^{-3}  \textrm{Mpc}^{-3} $, 
$M^*_{1600} = -21.16\pm 0.20$ and $\alpha = -1.91 \pm 0.09$. 
\citet{Bouwens15} reported that using just few fields of view, UVLF has a slightly non-Schechter-like form. 

\citet{Finkelstein14} worked also with HST, using the HUDF, CANDELS and GOODS fields, 
along with two of the Hubble Frontier Fields (HFF): deep parallel observations (unlensed 
fields) near the Abell 2744 and MACS J0416.1-2403 clusters. The HFF uses the ACS and 
the WFC3/IR with the same filters aforementioned but $z_{850}$. Total survey area is around 
$\sim300\textrm{arcmin}^2$, with 706 photometric candidates at redshift 6 defined as the interval $5.5<z<6.5$. 
Total volume of this study is around $ 8 \times 10^5 \textrm{Mpc}^3$. The Schechter function parameters they found are $\phi^* =(1.86_{-0.80}^{+0.94}) \times 10 ^{-4}  \textrm{Mpc}^{-3} $, 
$M^*_{1600} = -21.1_{-0.31}^{+0.25}$ and $\alpha = -2.02_{-0.10}^{+0.10}$. 



\citet{Willott13} presented the sixth release of the Canada-France-Hawaii 
Telescope Legacy Survey CFHTLS. The observations where performed over four 
separated fields covering a total area $\sim 4 \deg^2$ (a large area), it gives this 
survey great robustness. Optical observations used MegaCam  %%%%%%[CITE REQUIRED] 
with $u^* g' r' i' z'$ filters. The main selection criteria: all the 
objects must be brighter than magnitude $z' = 25.3$. The final 
number of LBGs founded was 40. Moreover, they get spectroscopic confirmation 
for 7 candidates using GMOS spectrograph on the Gemini Telescopes, which 
has a $\ll 5.5$-square arcmin field of view, giving a volume $\sim 1.09 \times 10^7 \textrm{Mpc}^3$. 
They show incompleteness in the sample due to 
foreground contamination and the detection algorithm; there is no warranty to 
have every object brighter than the limit magnitude on the faint limit. The full 
galaxy LF at $z=6$ cannot be obtained as in other studies. Nevertheless, this survey 
was focussed on the highly luminous LBGs. LF is calculated using the stepwise 
maximum likelihood method of \citet{Efstathiou88}, within 
magnitudes 
from $M_{1350} = -22.5$ up to $-20.5$. The luminosity function of $z=5.9$ shows 
an exponential decline at the bright end, where feedback processes and inefficient 
last cooling limites star forming in bright galaxies hosted in the most massive halos.

\citet{McLure09} build the luminosity function for $z=5$ and $z=6$ using data from two ground-based
 telescopes: the United Kingdom Infrared Telescope in the near-IR imaging and Subaru 
 Telescope for the optical imaging. They use the first data release of the UKIRT Infrared 
 DeepSky Survey Ultra Deep Survey (UDS), together with the Subaru XMM-Newton 
 Survey (SXDS). Total observed area is $0.63 \deg^2$ uniformly covered by booth catalogues.
The volume in this survey is $\sim 3 \times 10^6 \textrm{Mpc}^3$.
The UKIRT was equipped with the WFCAM using $J K$ filters. The Subaru was equipped 
with the Suprime-Cam with the $B V R i' z'$ filters. All candidates where brighter than 
$z'=26$. The UV rest frame magnitude range is $-22.4\leq M_{1500} \leq-20.6$. The LF 
was calculated using the maximum likelihood estimator of \citet{Schmidt68}. 
Their analysis gave a total number of 104 LBG candidates in the redshift range 
$5.7\leq z \leq 6.3$. LF was parameterized according to the Schechter function with 
$\phi^*  =(1.8 \pm 0.5) \times 10 ^{-3}  \textrm{Mpc}^{-3}$, $M^*_{1500} = -20.04\pm 0.12$ 
and $\alpha = -1.71 \pm 0.11$. 

The dataset was retrieved from \cite{McLure09} graph using
GAVO-DEXTER\footnote{\url{http://dc.zah.uni-heidelberg.de/dexter/ui/ui/custom}}.




\begin{figure}
\epsscale{1.00} 
\plotone{fig/observational_data.pdf}
\caption{Observational data from \cite{Bouwens15,McLure09}and \cite{Willott13}.}
\label{graph_observational_data}
\end{figure}


\section{Abundance Matching Methodology}
\label{sec:methodology}

The relation between UV luminosity and Star Formation Rate
\citep{Madau98,Kennicutt98} 
is given by:
\begin{equation}
 \textrm{SFR}\left(M_\odot \textrm{yr}^{-1}\right) 
      = 1.4 \times 10^{-28} L_{\nu} \left( \textrm{erg s}^{-1}\textrm{Hz}^{-1} 
	\right)
\end{equation}


  With Initial Mass Function (IMF) between $0.1 M_\odot$ 
 and $100 M_\odot$, in the range of $1250-2500 \mathring{\textrm{A}} $

The star forming rate will be:
  \begin{equation}
  SFR = k \times L_{0} M \left[ \left( \frac{M}{M_0}\right)^{-\beta} 
		   + \left( \frac{M}{M_0}\right)^{\gamma} 
               \right]^{-1}
  \end{equation}

The key element to connect SFR with halo mass from simulations is the observed UVLF. 
If there exist a function whom gives to each halo a UV luminosity then is posible to 
reproduce UVLF from DMH catalogs. The fitting parameters can be explored using a 
Markov Chain Monte-Carlo (MCMC) implementation. Once the parameters are found, 
the SFR-Halo Mass relation have been found.

We use the dark matter halo catalog from the Big MultiDark Plank 1 (MDPL) Simulation 
\citep{Klypin14} with 2013 Planck cosmology \citep{Planck1}. MDPL is quite similar to the 
Big Bolshoi ($1 \textrm{Gpc h}^{-1}$) \citep{Prada12} and 
its predecesor Bolshoi \citep{Klypin11} ($250 \textrm{Mpc h}^{-1}$), booth of them with 
WMAP5 cosmology, but MDPL has bigger mass resolution.  Those halo catalogs are 
available at the MultiDark Database\footnote{\url{http://www.multidark.org}} 
\citep{Riebe13}.

The MDPL run is a N-body dark matter only simulation based on the L-Gadget2 
code. The simulated volume is a cubic box of $1 \textrm{Gpc h}^{-1}$ side length. 
It has $3840^3$ dark matter particles of  $1.51\times 9
\textrm{M}_{\odot} \textrm{h}^{-1}$ mass each one. The 2013 Planck
cosmology is defined by the following parameters: $\Omega_M = 0.307$,
$\Omega_B = 0.048$, $\Omega_\Lambda = 0.730$, $\sigma_8 = 0.829$, $n_s
= 0.96$ and  $H_0 = 67.8$. The DMH Catalog at $z=6$ contains $\sim
10.9 \times 10^7$   halos, to avoid incompleteness in the low mass
end, halos with mass below   $10^{10.3} \textrm{M}_{\odot}
\textrm{h}^{-1}$ are rejected. In order to study how  
cosmic variance can affect measurements, the original catalog is
divided into 64 small   cubic boxes, with a similar volume to the
observations. 

The following treatment is applied to every one small box.

The halo abundance matching technique has been widely used \citep{Colin99, Kravtsov04}
resulting in good reproduction of the observed galaxy clustering. \citep{Conroy06, Lee09}. 
Here we use the simplest case with few premises: 
\begin{enumerate}
 \item Each halo in the catalog hosts one galaxy. There are not empty
halos, also none of halos has two or more galaxies.
 \item The UV luminosity of each galaxy is function of one variable; the mass of
the DMH in which is located. This function must be monotone, this guarantees that most 
massive halos will host the most luminous galaxies in the same volume.
\end{enumerate}

\subsection{Fitting Model}

The observed UV Luminosity Function has two slopes. To reproduce it we decide to use the
 four parameter model:
  \begin{equation}\label{eqn:luminosity}
  L_{\textrm{\small{UV}}} (M) = L_{0} M \left[ \left( \frac{M}{M_0}\right)^{-\beta} 
		   + \left( \frac{M}{M_0}\right)^{\gamma} 
               \right]^{-1}
  \end{equation}
where $M$ is the hosting DMH mass, $L_{0}$ is a normalization constant, $M_0$
is the critical mass where the luminosity function has a slope change, 
$\beta$ and $\gamma$ are the slopes. This equation has a similar fashion to the
mass to light relation \citep{vandenBosch03} and the mean relation between
stellar mas of a galaxy and the mass of its halo used by \citet{Moster10}.


\subsection{Dust Attenuation}
 
Dust in star forming galaxies absorbs part of the UV radiation and reemits on IR. The more 
massive is the galaxy then more dust contains and dust attenuation will be greater. The 
relation between dust attenuation and magnitude have been already studied, with this relation 
we can infer the dust-free UV luminosity of a galaxy from observations, resulting in  a more 
accurate inferred SFR.

The UV Spectral Slope $\beta$ was introduced by \citet{Meurer99} as a UV color to 
study dust attenuation in local starburst galaxies and extrapolating to high redshift galaxies. 
This index appears when a power law fitting is performed over the spectral flux $f$ 
as function of the wavelength $\lambda$;
\[ f \propto \lambda^\beta.\]
The relation for ultra-violet attenuation at $1600 \textrm{\AA}$ they found is:
\begin{equation}
A_{1600} = 4.43 + 1.99 \beta,
\end{equation}
with $A_{1600}$ in magnitude units. 

Due LBGs have more similar spectra properties to local starburst galaxies rather than AGNs 
for example, we can assume that local calibration of $\beta-A_{1600}$ can be applied also 
for LBGs,
%%% From Meurer99
``The main requirement is that the data include fluxes in two broad bands or coarse spectra 
covering the rest-frame UV.''


\citet{Bouwens12} uses the fluxes on different bands to estimate $\beta$ on each LBG 
candidate found with $z \sim 4-7$. After in redshift groups they found a linear relation 
between $\beta$ and the UV magnitude:
\begin{equation}
\langle \beta \rangle = \frac{d \beta}{d M_{UV}} \left( M_{UV,AB}+19.5 \right) 
                                   + \beta_{M_{UV}=-19.5}
\end{equation}
with $ \beta_{M_{UV}=-19.5} = -2.20$ and $d \beta/d M_{UV} = -0.21$ at $z=5.9$.

\citet{Smit12} used the aforementioned relations to infer the corrected luminosity functions,
i.e. dust-free luminosity functions, and the corrected SFR at $z=4$. 

Here we use the inverse relation, starting from the intrinsic or dust-free galaxy magnitude, 
obtaining the observed magnitude:
\begin{equation}
  M_{obs} = \begin{cases} 
                         \frac{M_{int}-4.616}{1.259}, &\mbox{if } A>0 \\
                         M_{int}, &\mbox{else}
                   \end{cases}.
\label{eqn. dust attenuation}
\end{equation}


\subsection{MCMC}
We used a Markov Chain Monte Carlo method to find the best parameters and its 
uncertainties over each one of the boxes and each observational dataset. The code was written
on Python using the Scipy, Numpy and Matplotlib libraries.

First of all, one of the four observational data sets (Bouwens, Finkelstein, Willott and McLure) 
is selected to be the model to fit.
From the whole simulated halo catalog, a subsample with cubic box shape of 
 $250\textrm{Mpc h}^{-1}$ length is selected. 
 
With an initial set of parameters $(\alpha, \gamma, M_0,$ and $ L_0)$ the UV luminosity
for the halo is calculated as function of his mass according to equation
 \ref{eqn:luminosity}. Then Luminosity is converted to magnitude units using:
  \[ M_{UV} = 51.82 - 2.5 \log_{10}(L_{UV}). \]
If we consider the dust attenuation im the equation \ref{eqn. dust attenuation} we have
a dust-corrected UV Luminosity Function.

The luminosity function is constructed as an histogram of the magnitudes normalized by 
the volume of the catalog. Each observed luminosity function has a different bin range.

Once having the LF, we compare our LF against observed LF. The error function we consider
is the sum of the square difference over each bin, divided by the observational data uncertain.
\[ \chi^2  = \sum_{i=o}^{n} \frac {\left( x_{i,obs} - x_{i,fit} \right)^2 }{2\sigma_i^2}\]
We worked on the logarithmic space of luminosities to have a good fitting on
six decades.

The likelihood will have this property:
\[ \mathcal{L} \propto \exp \left(  -\frac{\chi ^2}{n} \right) \], where $n$ is the number of degrees
of freedom. We have the maximum likelihood when the error is the minimum.

Each MCMC step will have have a small variation of the parameters, the UV luminosity and magnitude are 
calculated again for each halo, we have a new error $\chi_{new}^2$ and a new likelihood 
$ \mathcal{L}_{new} $.

Following the Metropolis method, we compare the new and the old likelihood:
\[ R =\frac{ \mathcal{L}_{new}  }{ \mathcal{L}_{old} }  = 
exp( \chi_{new}^2 - \chi_{old}^2 )
\]

If $  R \geq 1$, then we immediate accept the new set of parameters and start the next MCMC step.

Else, we have a chance to keep the new set. When $ R < 1$, we compare with a uniformly 
random number $p$ in the range $[0,1]$. if $ R>p$ we accept the new set of parameters and start
the next MCMC step. 
Else we reject the new set and start again with the old set.

We performed 10.000 effective MCMC steps, plus 1.000 thermalization steps over each box.
We repeated for the same box without consider dust attenuation.

Then we perform the same method over the 64 boxes and the resting three data sets.

The restrictions imposed over the parameters where $0 \leq \alpha \leq 2.0 $  and 
$\gamma \geq 0$.


Finally, the UV luminosity for each galaxy can be directly related with SFR according 
to \citet{Madau98}.
This model is accurate within the range of $10^8 - 10^9 \textrm{yr}$\citep{Kennicutt98}.
The relation between UV luminosity and Star Formation Rate \citep{Madau98} and \cite{Kennicutt98} 
is given by:
\begin{equation}
 \textrm{SFR}\left(M_\odot \textrm{yr}^{-1}\right) 
      = 1.4 \times 10^{-28} L_{\nu} \left( \textrm{erg s}^{-1}\textrm{Hz}^{-1} 
	\right)
\end{equation}

\subsection{The Luminosity Model}

In this model we have made two assumptions:
\begin{enumerate}
 \item Each halo in the catalog hosts one galaxy. There are not empty
halos, also none of halos has two or more galaxies.
 \item The UV luminosity of each galaxy is function of one variable: the mass of
the DMH in wich is located.
\end{enumerate}

The simplest relation we can have is a powerlaw:
 \begin{equation}
  L = L_0 M^\alpha
 \end{equation}
but has not well agreement with observed luminosity functions.

A better model is a four parameter function. Each galaxy has a luminosity given
by:
  \begin{equation}
  L = L_{0} M \left[ \left( \frac{M}{M_0}\right)^{-\beta} 
		   + \left( \frac{M}{M_0}\right)^{\gamma} 
               \right]^{-1}
  \end{equation}
where $M$ is the hosting DMH mass, $L_{0}$ is a normalization constant, $M_0$
is the critical mass where the luminosity function has a slope change, 
$\beta$ and $\gamma$ are the slopes. This equation has a similar fashion to the
mass to light relation \citep{vandenBosch03} and the mean relation between
stellar mas of a galaxy and the mass of its halo used by \cite{Moster10}.

There are more complex models\citep{Lee09} that includes a random behavior:
galaxies has not synchronization on the beginning of star forming stage, also
this stage may be time limited. This is called duty cycle. It is probable to
have in the observations some invisible galaxies in the UV continuum due their
duty cycle may has not started as well it may ended. Also may be present a
normal distribution of the luminosity around the expected values.

\section{Results}
\label{sec:results}

\subsection{Willott}

We performed MCMC runnings with 100.000 steps over the 64 small 
cubic boxes (of $250$ \hMpc side length) fitting the LF to the 
Willott observational data, comparing the two cases; with and 
without dust attenuation. (figures \ref{fig:OD1_MCMC_best_steps_w} and 
\ref{fig:OD1_MCMC_best_steps_wo}). 
We used the Likelihood Ratio criterion ($\mathcal{LR}= 0.5$) to
define the $1\sigma$ confidence interval for our parameters.

$M_0$ is quite similar in booth cases (they are compatible within 
the error bars), the turnover point corresponds to the same mass.
$\gamma$ and $L_0$ shown a significative difference in booth cases.
$\beta$ is hard to constraint in booth cases. 
The parameer was limited to vary in the range form 0.0 to 1.6. 
$1\sigma$ region covers the whole range.

The UV luminosity model (eqn. \ref{eqn:luminosity}) that we have 
chosen can be divided in two regimes; high mass regime 
(with $M > M_0$) and low mass regime (with $M < M_0$).

The observational dataset from Willott are in the high mass regime 
with one point in the low mass regime. It makes makes hard to impose
restrictions over $\beta$, but the other three parameters can be
well defined.

We also compare the likelihood of the two cases on each individual 
small box. The Dust Attenuation model is more acurrate than the
No-Dust Attenuation model in most of the cases as is shown in the figure 
\ref{fig:OD1_chi2_comparison}.

To study cosmic variance effects, we compared the best fit parameters
of each box and its likelihood value. We found that cosmic variance
effects are less significative in best fit parameters than MCMC parameter
estimation itself. 



\begin{table}
\begin{center}
\begin{tabular}{ccccc}\hline\hline
Parameter  & Mean value & Cosmic Variance & MCMC error\\\hline
$\log_{10}(L_0)$   & 18.067   & 0.029   & 0.116 \\
$\log_{10}(M_0)$  & 11.190   & 0.072   & 0.729 \\
$\beta$                  & 1.4801  & 0.175   & 1.299 \\
$\gamma$             & 0.40284 & 0.058  & 0.3147\\
\hline\hline
\end{tabular}
\caption{  Average Best fit parameters to the Willott data in the Dust Attenuation Model.} 
\label{table:Willott_best_fit_parameters}
\end{center}
\end{table}

\begin{table}
\begin{center}
\begin{tabular}{ccccc}\hline\hline
Parameter  & Mean value & Cosmic Variance & MCMC error\\\hline
$\log_{10}(L_0)$   & 17.807   & 0.020   & 0.150 \\
$\log_{10}(M_0)$  & 11.119    & 0.060   & 0.642 \\
$\beta$                  &  1.437    & 0.178   & 1.374 \\
$\gamma$             &   0.518   & 0.046  & 0.246\\
\hline\hline
\end{tabular}
\caption{  Average Best fit parameters to the Willott data in the No-Dust Attenuation Model.} 
\label{table:Willott_best_fit_parameters_no_dust}
\end{center}
\end{table}



\begin{figure}
\epsscale{1.00}
\plotone{fig/OD1_chi2_comparison.pdf}
\caption{Best fit to Willott comparison between the two models. Each point represent $\chi^2$ 
calculated over each small-box. The solid line represents the ratio 1:1 }
\label{fig:OD1_chi2_comparison}
\end{figure}

\begin{figure*}
\epsscale{1.00}
\plotone{fig/OD1_w_dust_box_0.png} % pdf available
\caption{Parameter dispersion fitting the Willott with the Dust Attenuation model. 
$1\sigma$ is defined by the likelihood ratio between 0.0 (red) and 0.5 (green)}
\label{fig:OD1_MCMC_best_steps_w}
\end{figure*}

\begin{figure*}
\epsscale{1.00}
\plotone{fig/OD1_wo_dust_box_0.png} % pdf available
\caption{Parameter dispersion fitting the Willott with the No-Dust Attenuation model. 
$1\sigma$ is defined by the likelihood ratio between 0.0 (red) and 0.5 (green)}
\label{fig:OD1_MCMC_best_steps_wo}
\end{figure*}





\begin{figure}
\epsscale{1.00}
\plotone{fig/OD1_CosmicVar1_w_dust.pdf}
\caption{Individual small-box parameter estimation with dust attenuation: Best fit values 
to the Willott data set
(solid line) and $1\sigma$ confidence interval using likelihood ratio $\mathcal{LR}= 0.5$. 
The $x$ axis corresponds to the box number.}
\label{fig:OD1_individual_box_results_w}
\end{figure}


\begin{figure}
\epsscale{1.00}
\plotone{fig/OD1_CosmicVar1_wo_dust.pdf}
\caption{Individual small-box parameter estimation with no-dust attenuation: Best fit values 
to the Willott data set
(solid line) and $1\sigma$ confidence interval using likelihood ratio $\mathcal{LR}= 0.5$. 
The $x$ axis corresponds to the box number.}
\label{fig:OD1_individual_box_results_wo}
\end{figure}


\begin{figure}
\epsscale{1.00}
\plotone{fig/OD1_CosmicVar2_w_dust.pdf}
\caption{Best fit parameter distribution due cosmic variance with Willott data in the dust
 attenuation model.}
\label{fig:OD1_CosmicVar2_w_dust}
\end{figure}


\begin{figure}
\epsscale{1.00}
\plotone{fig/OD1_CosmicVar2_wo_dust.pdf}
\caption{Best fit parameter distribution due cosmic variance with Willott data in the no-dust
 attenuation model.}
\label{fig:OD1_CosmicVar2_wo_dust}
\end{figure}

\begin{figure}
\epsscale{1.00}
\plotone{fig/SFR_DMHM_wo.pdf}
\caption{Star formation rate as function of the dark matter halo mass without dust attenuation. 
Solid lines represents the mean SFR value over the small boxes within $50\%$  shaded region. 
Comparison with the GALFORM semi-analitical model \citep{Gonzalez14} and a implementation
of abundance matching model \citep{Behroozi13}. }
\label{graph_SFR_DMHM_wo}
\end{figure}

\begin{figure}
\epsscale{1.00}
\plotone{fig/SFR_DMHM_w.pdf}
\caption{Star formation rate as function of the dark matter halo mass with dust attenuation. 
Solid lines represents the mean SFR value over the small boxes within $50\%$  shaded region.
Comparison with the GALFORM semi-analitical model \citep{Gonzalez14} and a implementation
of abundance matching model \citep{Behroozi13}. }
\label{graph_SFR_DMHM_w}
\end{figure}



\section{Discussion}
\label{sec:discussion}


\section{Conclusions}
\label{sec:conclusions}


\section*{Acknowledgments}
Acknowledgments...\\


\bibliography{references}

\end{document}

