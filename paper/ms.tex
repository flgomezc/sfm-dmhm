\documentclass{emulateapj}
\submitted{{\it Submitted for publication in ApJ}}
\usepackage{multirow,color,wrapfig,ulem}
\usepackage {graphicx}
\usepackage{graphics}
\usepackage{amsmath}
\usepackage[dvips]{epsfig}
\bibliographystyle{apj}
\newcommand{\avg}[1]{\langle{#1}\rangle}  
\newcommand{\nscatt}{\langle N_{\rm  scatt}\rangle}
\newcommand{\ly}{{\ifmmode{{\rm Ly}\alpha~}\else{Ly$\alpha$~}\fi}}
\newcommand{\hMpc}{{\ifmmode{h^{-1}{\rm Mpc}}\else{$h^{-1}$Mpc }\fi}}   
\newcommand{\hGpc}{{\ifmmode{h^{-1}{\rm Gpc}}\else{$h^{-1}$Gpc }\fi}}   
\newcommand{\hmpc}{{\ifmmode{h^{-1}{\rm Mpc}}\else{$h^{-1}$Mpc }\fi}}  
\newcommand{\hkpc}{{\ifmmode{h^{-1}{\rm kpc}}\else{$h^{-1}$kpc }\fi}}  
\newcommand{\hMsun}{{\ifmmode{h^{-1}{\rm
        {M_{\odot}}}}\else{$h^{-1}{\rm{M_{\odot}}}$}\fi}}   
\newcommand{\hmsun}{{\ifmmode{h^{-1}{\rm
        {M_{\odot}}}}\else{$h^{-1}{\rm{M_{\odot}}}$}\fi}}   
\newcommand{\Msun}{{\ifmmode{{\rm {M_{\odot}}}}\else{${\rm{M_{\odot}}}$}\fi}}  
\newcommand{\msun}{{\ifmmode{{\rm {M_{\odot}}}}\else{${\rm{M_{\odot}}}$}\fi}}  
\newcommand{\lya}{{Lyman $\alpha$~}}
\newcommand{\clara}{{\texttt{CLARA}}~}
\newcommand{\rand}{{\ifmmode{{\mathcal{R}}}\else{${\mathcal{R}}$ }\fi}}  
\newcommand{\hs}{{\hspace{1mm}}}  
\newcommand{\kms}{{\ifmmode{{\mathrm{\,km\ s}^{-1}}}\else{\,km~s$^{-1}$}\fi}}
% definition to produce a "less than or similar to" symbol
\def\lsim{~\rlap{$<$}{\lower 1.0ex\hbox{$\sim$}}}
% definition to produce a "greater than or similar to" symbol
\def\gsim{~\rlap{$>$}{\lower 1.0ex\hbox{$\sim$}}}
%@arxiver{fig3.pdf,fig11a.pdf, fig11b.pdf} 
\begin{document}



\title{The connection between star formation rate and dark matter halo
  mass in the epoch of reionization}  
\shorttitle{SFR and DM halo mass during reionization}

\shortauthors{Gomez-Cortes, Forero-Romero}

\author{Felipe L. Gomez-Cortes, Jaime E. Forero-Romero}  
\affil{Departamento de F\'{i}sica, Universidad de los Andes, Cra. 1
No. 18A-10, Edificio Ip, Bogot\'a, Colombia}
\email{fl.gomez10@uniandes.edu.co}
\email{je.forero@uniandes.edu.co}

\keywords{galaxies: high-redshift --- methods: numerical} 
\begin{abstract}
We present updated constraints on the relationship between the star
formation rate and dark matter halo mass at redshift $z\sim 6$.
The observational basis for our work is the restframe UV luminosity
function data obtained with HST, CFHTLS, Subary and UKIRT.
The constraints are based on an abundance matching methodology to the
observational data using cosmological N-body simulations.
The relationship between halo mass and star formation rate follows a
double power law.
We also take into account the influence on the results of the dust extinction
scaling derived from observations by Bouwens et al. (2012).
Taking advantage of a wide dynamical range in observations and
simulations we manage to constrain the free parameters of our model
for halos in the mass range $10^{10}\Msun < M_{h}<10^{13}\Msun$. 
We find that including dust extinction improves the the match with
observations.
Finally, we compare these results against the prediction of abundance matching
methods (to the stellar mass), a semi-analytic model of galaxy
formation (GALFORM) and a hydrodynamical simulation (Illustris).
\end{abstract}


\section{Introduction}
\label{sec:intro}


All magnitudes are in AB system.
\section{Observational Constraints}
\label{sec:theo}

We use information compiled in four different publications. 
All of them select galaxy candidates at $z\sim 6$ using the drop-out
technique \citep{Steidel96}.  
In what follows we describe the relevant details of each reference.


\subsection{Bouwens et al. 2015}
\citet{Bouwens15} presented results from a compilation of
observations taken with the Advanced Camera for Surveys (ACS) and
near-infrared Wide Field Camera 3 (WFC3/IR) since 2002 through 2012. 
The study inclues the following fields of view. XDF,
HUDF09-1, HUDF09-2, CANDELS-S/Deep,  CANDELS-S/Wide, ERS,
CANDELS-N/Deep, CANDELS-N/Wide, CANDELS-UDS,  CANDELS-COSMOS and
CANDELS-EGS.
%The information in that study comes from the ACS and WFC3/IR using the
%filter $B_{435}$, $V_{606}$, $ i_{814}$, $ z_{850}$, $ I_{814}$, $Y_{098}$, $Y_{105}$,
%$J{125}$,  $JH_{140}$ and $H_{160}$.

%with areas of 4.7, 4.7, 4.7, 64.5, 34.2,  40.5, 62.9,
%60.9, 151.2, 151.9 and 150.7 arcmin$^2$ respectively.  
The total survey area is $740.8$ arcmin$^2$ over five different lines
of sight, with a total estimated volume of  $1.8 \times 10^6
\textrm{Mpc}^3$ comoving. 
The limiting magnitude ranges between $\sim27.5$mag in CANDELS-EGS and
$\sim30 $mag in  the  deepest field (XDF). 
The total number of $z=6$ LBG candidates is $940$, most of 
them in the faint end of the LF given the relativeley small survey volume.
The restframe UV magnitudes for these candidates are in the range.
$-22.52\leq M_{1600} \leq -16.77$.
The estimated Schechter parameters:
$\phi^* =(0.33_{-0.10}^{+0.15}) \times 10 ^{-3}  \textrm{Mpc}^{-3} $, 
$M^*_{1600} = -21.16\pm 0.20$ and $\alpha = -1.91 \pm 0.09$. 
%\citet{Bouwens15} notes that only using few fields of view can yield a
%LF with a deviations from the Schechter shape.


\subsection{Finkelstein et al. 2014}
\citet{Finkelstein14} worked also with HST data.
They used results from the HUDF, CANDELS and GOODS  fields, along with
two of the Hubble Frontier Fields (HFF) of deep parallel  observations
(unlensed fields) near the Abell 2744 and MACS J0416.1-2403  clusters. 
The HFF uses the ACS and the WFC3/IR with the same filters
aforementioned except $z_{850}$. 
The total survey area is around   $\sim300\textrm{arcmin}^2$ with a
total estimated volume of $8\times 10^5$ Mpc$^{-3}$.
There are 706 photometric candidates at redshift 6  defined as the
interval $5.5<z<6.5$. 
The Schechter function parameters estimated for this data set 
are $\phi^* =(1.86_{-0.80}^{+0.94}) \times 10 ^{-4}  \textrm{Mpc}^{-3} $, 
$M^*_{1600} = -21.1_{-0.31}^{+0.25}$ and $\alpha = -2.02_{-0.10}^{+0.10}$. 
 
\subsection{Willott et al. 2013}
\citet{Willott13} presented results form the sixth release of the Canada-France-Hawaii 
Telescope Legacy Survey (CFHTLS). 
The observations where performed over four separated fields covering a
total area $\sim 4 \deg^2$ giving a survey volume of $\sim 1 \times
10^7 \textrm{Mpc}^3$, which is over one order of magnitude larger than
the compilations by \citet{Bouwens15} and \citet{Finkelstein14} 

They performed otical observations with MegaCam.
Their main selection criteria was that all objects must be brighter
than magnitude $z' = 25.3$. 
The final  number of LBGs founded was 40. 
Moreover, they get spectroscopic confirmation 
for 7 candidates using GMOS spectrograph on the Gemini Telescopes, which 
has a $\ll 5.5$-square arcmin field of view, 

%They show incompleteness in the sample due to 
%foreground contamination and the detection algorithm; there is no warranty to 
%have every object brighter than the limit magnitude on the faint
%limit. 
The survey was focused on the most luminous LBGs. 
The full LF at $z=6$ cannot be obtained as in other studies, however
its large volume allows constraints on the bright end.
The LF Schechter parameters are calculated in the magnitud range 
$-20.5 > M_{1350} > -22.5$.

\subsection{McLure et al. 2009}

\citet{McLure09} used data obtained with the 
the United Kingdom Infrared Telescope (UKIRT) in the near-IR imaging
and Subaru Telescope for the optical imaging. 
They covered $2268$ arcmin$^{2}$  for a volume $\sim 3 \times 10^6 \textrm{Mpc}^3$.
The UKIRT was equipped with the WFCAM using $J K$ filters, while Subaru was 
equipped with the Suprime-Cam with the $B V R i' z'$ filters. 
All candidates  where brighter than $z'=26$. 
The UV rest frame magnitude range is 
$-22.4\leq M_{1500} \leq-20.6$. 
The LF was calculated using the maximum 
likelihood estimator of \citet{Schmidt68}. 
Their analysis gave a total number of 
$104$ LBG candidates in the redshift range $5.7\leq z \leq 6.3$. 
They reported the following values for the Schechter function
$\phi^* =(1.8\pm 0.5)\times 10 ^{-3} \textrm{Mpc}^{-3}$, 
$M^*_{1500} = -20.04\pm 0.12$ and $\alpha = -1.71 \pm 0.11$. 


\begin{figure}
\epsscale{1.00} 
\plotone{fig/observational_data.pdf}
\caption{Observational data from \cite{Bouwens15,McLure09}and \cite{Willott13}.}
\label{graph_observational_data}
\end{figure}

\subsection{Dust Attenuation}
\label{subsec:dust}
Dust in star forming galaxies can attenuate the UV intrinsic luminosity.
Although relatively uncertain compared to studies in the local
universe, there are constraints on the extinction level in LBGs at
$z=6$.

These studies are based on the UV Spectral slope $\beta$ correlation
with extinction. 
This index can be defined approximating by a power-law
the shape  the spectral flux $f$ as function of the wavelength
$\lambda f \propto \lambda^\beta$.  
The relation for attenuation at $1600 \textrm{\AA}$ is $A_{1600} =
4.43 + 1.99 \beta$, with $A_{1600}$ in magnitude units
\citet{Meurer99}.  

\citet{Bouwens12} uses the fluxes on different bands to estimate $\beta$ on each LBG 
candidate found with $z \sim 4-7$. 
At $z=6$ they found a linear relation beteen between $\beta$ and
$M_{\rm UV}$:

\begin{equation}
\langle \beta \rangle = \frac{d \beta}{d M_{UV}} \left( M_{UV,AB}+19.5 \right) 
                                   + \beta_{M_{UV}=-19.5},
\end{equation}
with $ \beta_{M_{UV}=-19.5} = -2.20$ and $d \beta/d M_{UV} = -0.21$ at
$z=5.9$. 
We use these relationships to modify the instrinsic UV luminosity
values.

%Here we use the inverse relation, starting from the intrinsic or dust-free galaxy magnitude, 
%obtaining the observed magnitude:
%\begin{equation}
%  M_{obs} = \begin{cases} 
%                         \frac{M_{int}-4.616}{1.259}, &\mbox{if } A>0 \\
%                         M_{int}, &\mbox{else}
%                   \end{cases}.
%label{eqn. dust attenuation}
%end{equation}



\section{Theoretical, Numerical and Statistical Framework}
\label{sec:methodology}


\subsection{The Abundance Matching Methodology}
We use an abundance matching approach to find the relationship between
dark matter halo mass and star formation rate. 
This approach has been used to link stellar and dark matter masses. 
Here we extend it to constraint star formation rate properties at high
redshift. 

The starting point for this method is a population of dark matter halos.
To each halo we assign a UV luminosity according to the following
parameterization.
  \begin{equation}\label{eqn:luminosity}
  L_{\textrm{\small{UV}}} (M) = L_{0} M \left[ \left( \frac{M}{M_0}\right)^{-\beta} 
		   + \left( \frac{M}{M_0}\right)^{\gamma} 
               \right]^{-1},
  \end{equation}
where $M$ is the DM mass, $L_{0}$ is a normalization constant, $M_0$
is the critical mass where the luminosity function has a slope change, 
$\beta$ and $\gamma$ are two slopes corresponding to the faint and
bright end, respectively. 
We now that this equation has the same functional dependence suggested
 by \citet{Moster10} to link stellar and dark matter masses.

Once each halo has a UV luminosity we also have the option to include
an extinction correction (as described in \SS \ref{subsec:dust})in
order to modify this intrinsic luminosity value. 
From these UV values we build the LF.
The free parameters in Eq. (\ref{eqn:luminosity}) are determined by
requiring that the abudance matching LF follows the observational
constraints. 
Then we use the following relationship between UV luminosity and Star Formation Rate
\citep{Madau98,Kennicutt98} 
\begin{equation}
 \textrm{SFR}\left(M_\odot \textrm{yr}^{-1}\right) 
      = 1.4 \times 10^{-28} L_{\nu} \left( \textrm{erg s}^{-1}\textrm{Hz}^{-1} 
	\right), 
\end{equation}
to finally link SFR with DM halo mass.



\subsection{N-body Simulations and Halo Catalogs}

Cosmological N-body simulations are the source of the dark matter halo
populations.
We use two different simulations to cover the wide dynamical range
explored by the observations: Big MultiDark Planck (MDPL) and Planck
Bolshoi (P-Bolshoi). 
Both simulations use 2013 Planck cosmology  defined by the following
parameters: $\Omega_M = 0.307$, $\Omega_B = 0.048$, $\Omega_\Lambda =
0.730$, $\sigma_8 = 0.829$, $n_s = 0.96$ and  $H_0 = 67.8$.  

The MDPL run is a N-body dark matter only simulation based on the L-Gadget2 
code. 
The simulated volume is a cubic box of $1 \textrm{Gpc h}^{-1}$ on a side.
It has $3840^3$ dark matter particles mass of $1.51\times 9
\textrm{M}_{\odot} \textrm{h}^{-1}$.
The DMH Catalog at $z=6$ contains $\sim 10.9 \times 10^7$   halos, to
avoid incompleteness in the low mass end, halos with mass below
$10^{10.3} \textrm{M}_{\odot} \textrm{h}^{-1}$ are rejected. 
We split this large volume into 64 smaller cubic boxes of $250$\hMpc
on a side (similar volume to the observations by \citet{Willott13}) to
study the influence of cosmic variance. 

The P-Bolshoi simulation...


We obtained the data from the public
database\footnote{\url{http://www.multidark.org}} \citep{Riebe13}. 



\subsection{Constraining the Free Parameters}


We use a Markov Chain Monte Carlo (MCMC) methodology to constraint the 
the free parameters in Eq. (\ref{eqn:luminosity}). 
We use a Metropolis-Hastings algorithm to build the MC chains on the
four parameters. 
We performed $10^5$ MCMC steps with $10^4$ additional initial burn-out
iterations.


To explore the likelihood ${\mathcal L}\propto \exp{(-\chi^2)}$we
consider the following $\chi^2$. 
\begin{equation} 
\chi^2  = \sum_{i=o}^{n} \frac {\left( \log\Phi_{i,obs} - \log\Phi_{i,th} \right)^2
}{2\sigma_i^2}, 
\end{equation}
defined over each LF magnitude bin.

We treat each dataset (Bouwens, Finkelstein, Willott and McLure)
separately to join at the end the posteriors. 
The size and halo resolution of the computational boxes is coarse for
the case of Willott and fine for the other datasets.
We perform two kinds of MCMC explorations with and without considering
dust extinctions.
We impose the following priors over the parameters $0 \leq \alpha \leq
2.0 $  and  $\gamma \geq 0$. 
 

\section{Results}
\label{sec:results}

\subsection{Willott}

We use 64 cubic boxes of $250$\hMpc on a side to fit the
LF data from \citet{Willott13}.
We use the Likelihood Ratio criterion ($\mathcal{LR}= 0.5$) to
define the $1\sigma$ confidence interval for our parameters.

$M_0$ does not change with and without extinction.
cases (they are compatible within 
the error bars), the turnover point corresponds to the same mass.
$\gamma$ and $L_0$ shown a significative difference in booth cases.
$\beta$ is hard to constraint in booth cases.  
The parameer was limited to vary in the range form 0.0 to 1.6. 
$1\sigma$ region covers the whole range.

The UV luminosity model (eqn. \ref{eqn:luminosity}) that we have 
chosen can be divided in two regimes; high mass regime 
(with $M > M_0$) and low mass regime (with $M < M_0$).

The observational dataset from Willott are in the high mass regime 
with one point in the low mass regime. It makes makes hard to impose
restrictions over $\beta$, but the other three parameters can be
well defined.

We also compare the likelihood of the two cases on each individual 
small box. The Dust Attenuation model is more acurrate than the
No-Dust Attenuation model in most of the cases as is shown in the figure 
\ref{fig:OD1_chi2_comparison}.

To study cosmic variance effects, we compared the best fit parameters
of each box and its likelihood value. We found that cosmic variance
effects are less significative in best fit parameters than MCMC parameter
estimation itself. 

\begin{table}
\begin{center}
\begin{tabular}{cccccc}\hline\hline
 & \multicolumn{2}{c}{Dust Att.} & \multicolumn{2}{c}{No-Dust Att.}\\
Parameter                  & MCMC                  & C.Var. & MCMC                    & C.Var. \\\hline
$\log_{10}(L_0/L_{\odot})$ & $18.07_{-0.06}^{+0.10}$ & 0.03 & $17.81_{-0.08}^{+0.12}$ & 0.02  \\
$\log_{10}(M_0/M_{\odot})$ & $11.19_{-0.03}^{+0.67}$ & 0.07 & $11.12_{-0.24}^{+0.58}$ & 0.06  \\
$\beta$                    & $1.480_{-1.28}^{+0.11}$ & 0.18 & $ 1.44_{-1.35}^{+0.16}$ & 0.18  \\
$\gamma\times0.1$          & $4.028_{-0.82}^{+3.01}$ & 0.58 & $ 5.18_{-0.59}^{+2.36}$ & 0.46  \\
\hline\hline
\end{tabular}
\caption{ Best fit parameters to the Willott data with and without 
dust attenuation over 64 small boxes. Mean best value estimated 
with MCMC, mean $1\sigma$ confidence interval and Cosmic Variance
(C.Var.)
} 
\label{table:Willott_best_fit_parameters_no_dust}
\end{center}
\end{table}


\begin{figure}
\epsscale{1.00}
\plotone{fig/OD1_chi2_comparison.pdf}
\caption{Best fit to Willott comparison between the two models. Each point represent $\chi^2$ 
calculated over each small-box. The solid line represents the ratio 1:1 }
\label{fig:OD1_chi2_comparison}
\end{figure}

\begin{figure}
\epsscale{1.00}
\plotone{fig/OD1_w_dust_box_0.png} % pdf available
\caption{Parameter dispersion fitting the Willott with the Dust Attenuation model. 
$1\sigma$ is defined by the likelihood ratio between 0.0 (red) and 0.5 (green)}
\label{fig:OD1_MCMC_best_steps_w}
\end{figure}

\begin{figure}
\epsscale{1.00}
\plotone{fig/OD1_wo_dust_box_0.png} % pdf available
\caption{Parameter dispersion fitting the Willott with the No-Dust Attenuation model. 
$1\sigma$ is defined by the likelihood ratio between 0.0 (red) and 0.5 (green)}
\label{fig:OD1_MCMC_best_steps_wo}
\end{figure}





\begin{figure}
\epsscale{1.00}
\plotone{fig/OD1_CosmicVar1_w_dust.pdf}
\caption{Individual small-box parameter estimation with dust attenuation: Best fit values 
to the Willott data set
(solid line) and $1\sigma$ confidence interval using likelihood ratio $\mathcal{LR}= 0.5$. 
The $x$ axis corresponds to the box number.}
\label{fig:OD1_individual_box_results_w}
\end{figure}


\begin{figure}
\epsscale{1.00}
\plotone{fig/OD1_CosmicVar1_wo_dust.pdf}
\caption{Individual small-box parameter estimation with no-dust attenuation: Best fit values 
to the Willott data set
(solid line) and $1\sigma$ confidence interval using likelihood ratio $\mathcal{LR}= 0.5$. 
The $x$ axis corresponds to the box number.}
\label{fig:OD1_individual_box_results_wo}
\end{figure}


\begin{figure}
\epsscale{1.00}
\plotone{fig/OD1_CosmicVar2_w_dust.pdf}
\caption{Best fit parameter distribution due cosmic variance with Willott data in the dust
 attenuation model.}
\label{fig:OD1_CosmicVar2_w_dust}
\end{figure}


\begin{figure}
\epsscale{1.00}
\plotone{fig/OD1_CosmicVar2_wo_dust.pdf}
\caption{Best fit parameter distribution due cosmic variance with Willott data in the no-dust
 attenuation model.}
\label{fig:OD1_CosmicVar2_wo_dust}
\end{figure}

\begin{figure}
\epsscale{1.00}
\plotone{fig/SFR_DMHM_wo.pdf}
\caption{Star formation rate as function of the dark matter halo mass without dust attenuation. 
Solid lines represents the mean SFR value over the small boxes within $50\%$  shaded region. 
Comparison with the GALFORM semi-analitical model \citep{Gonzalez14} and a implementation
of abundance matching model \citep{Behroozi13}. }
\label{fig:SFR_DMHM_wo}
\end{figure}

\begin{figure}
\epsscale{1.00}
\plotone{fig/SFR_DMHM_w.pdf}
\caption{Star formation rate as function of the dark matter halo mass with dust attenuation. 
Solid lines represents the mean SFR value over the small boxes within $50\%$  shaded region.
Comparison with the GALFORM semi-analitical model \citep{Gonzalez14} and a implementation
of abundance matching model \citep{Behroozi13}. }
\label{fig:_SFR_DMHM_w}
\end{figure}



\section{Discussion}
\label{sec:discussion}


\section{Conclusions}
\label{sec:conclusions}


\section*{Acknowledgments}
Acknowledgments...\\

The observational datasets were retrieved using
GAVO-DEXTER\footnote{\url{http://dc.zah.uni-heidelberg.de/dexter/ui/ui/custom}}. 


\bibliography{references}

\end{document}










\begin{table*}
\begin{center}
\begin{tabular}{cccccc}\hline\hline
 & \multicolumn{2}{c}{Dust Attenuation} & \multicolumn{2}{c}{No-Dust Attenuation}\\
Parameter  & MCMC & Cosmic Variance & MCMC & Cosmic Variance \\\hline
$\log_{10}(L_0)$   & $18.067_{-0.059}^{+0.096}$   & 0.029 & $17.807_{-0.083}^{+0.115}$   & 0.020  \\
$\log_{10}(M_0)$   & $11.190_{-0.026}^{+0.674}$   & 0.072 & $11.119_{-0.237}^{+0.582}$   & 0.060  \\
$\beta$            & $1.4801_{-1.276}^{+0.114}$   & 0.175 & $ 1.437_{-1.348}^{+0.157}$   & 0.178  \\
$\gamma\times0.1$  & $4.0284_{-0.822}^{+3.010}$   & 0.576 & $ 5.182_{-0.588}^{+2.366}$  & 0.461  \\
\hline\hline
\end{tabular}
\caption{  Average Best fit parameters to the Willott data with and without dust attenuation.} 
\label{table:Willott_best_fit_parameters_no_dust}
\end{center}
\end{table*}



