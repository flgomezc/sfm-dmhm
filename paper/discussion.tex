\section{Discussion}

\begin{figure}
\epsscale{1.00}
\plotone{fig/cosmic_variance.pdf}
\caption{Cosmic Variance: The Luminosity Function is made using the DMH catalog 
from the full box and the set of parameters from the small boxes.}
\label{graph_cosmic_variance}
\end{figure}

\citep{lundgren14} SFR evolution from $z=1$ to $6$

\citep{bouwens14} HST Legacy

\citep{jiang11} Keck pectroscopy


``Fig. 16.— Updated Determinations of the derived SFR (left axis) and U V
luminosity (right axis) densities versus redshift (5.4). The
left axis gives the SFR densities we would infer from the measured luminosity
densities, assuming the Madau et al. (1998) conversion
factor relevant for star-forming galaxies with ages of 108 yr (see also
Kennicutt 1998). The right axis gives the U V luminosities we infer
integrating the present and published LFs to a faint-end limit of −17 mag (0.03
L∗ )''BOUWENS 2014 UV LF

\begin{figure}
\epsscale{1.00}
\plotone{fig/scattering_plots.png}
\caption{Covariance of parameters for one small box}
\label{graph_scattering_plots}
\end{figure}




\begin{figure}
\epsscale{1.00}
\plotone{fig/SFR_DMHM_wo.pdf}
\caption{Star formation rate as function of the dark matter halo mass without dust attenuation. 
Solid lines represents the mean SFR value over the small boxes within $50\%$  shaded region. }
\label{graph_SFR_DMHM_wo}
\end{figure}

\begin{figure}
\epsscale{1.00}
\plotone{fig/SFR_DMHM_w.pdf}
\caption{Star formation rate as function of the dark matter halo mass with dust attenuation. 
Solid lines represents the mean SFR value over the small boxes within $50\%$  shaded region. }
\label{graph_SFR_DMHM_w}
\end{figure}