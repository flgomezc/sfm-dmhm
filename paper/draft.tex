\documentclass[manuscript]{aastex}
%\documentclass[preprint2]{aastex}
%\usepackage{graphicx}
%\usepackage{graphics}
%\usepackage[dvips]{epsfig}
\usepackage{epstopdf}

\newcommand{\vdag}{(v)^\dagger}
\newcommand{\myemail}{skywalker@galaxy.far.far.away}

\shorttitle{A Connection Between Star Formation Rate and Dark Matter Halos at Z 
\~ 6}
\shortauthors{G\'omez-Cort\'es}

\begin{document}

\title{A Connection Between Star Formation Rate and Dark Matter Halos at 
$Z\sim6$ In 2013 Planck Cosmology.}

\author{F.L. G\'omez-Cort\'es\altaffilmark{1} }
\affil{Departamento de Física, Universidad de los Andes, Colombia}

\begin{abstract}
This work relates baryonic matter and dark matter at redshift $z=5.9$ using 
observational data from CFHTLS \citep{willott13}, HUDF09 
\citep{bouwens06,bouwens12}, UKIDSS and SDXS \citep{mclure09}, and results of 
the Multidark Simulation \citep{riebe13} in a cubic box of $1000 \textrm{Mpc 
h}^{-1}$ length with 2013 Planck Cosmology. The Luminosity Function (LF) is 
fitted via four parameters with the Markov Chain Monte Carlo method. The 
relationship between the Dark Matter Halos Mass and Star Formation Rate is 
obtained using the relationship between the UV  continuum (from the fitted LF) 
and Star Formation Rate (SFR) by \cite{kennicutt98}.Cosmic variance effects are 
studied on smaller boxes of $250 \textrm{Mpc h}^{-1}$ length.

%Luminosity Function - Dark Matter Halos, Star Formation Rate - Dark Matter 
Halos.
\end{abstract}
\keywords{Dark Matter, LF, SFR, High Redshift Galaxies, Reionization}

\section{Introduction}

Dark matter is a significative component of the universe.

\citep{trimble87}

Hierarchy Structure Evolution: Early formation of small structures
merging on major structures after.

From simulations, Halo Mass Function as function of redshift (or time).

Star formation rate as function of time. Peak at $z\sim 2$.

Main Objetive: Reproduce the observed luminosity function at redshift $z=5.9$
from a DMH catalog from simulations. 

  \subsection{Halo Mass Function (HMF)}
  
  The HMF is created using the DMH catalog from the Multidark Database 
\citep{riebe13}.
  
  
  \subsection{Cosmic Variance}
  
  Is important to know and understand the existence of local fluctuations   or 
inhomogeneities due the observed scale.  For this part is necesary to study the 
HMF on smaller boxes

\section{Relationship between Galaxy Luminosity Function (GLF) and
  Dark Matter Halo (DMH) mass}

  In one hand we have observational results: the GLF for star-forming galaxies 
at high redshift (z=5.9)\citep{bouwens06,willott13}  expressed in terms of 
magnitude in the ultraviolet range (M1350). In the other hand we have a DMH  
catalog, result from the Multidark Simulation with Plank Cosmology. 
%\citep{Multidark}

  A $1000 \textrm{ Mpc} ^3 \textrm{h}^{−3}$ cubic box containing near to  
$11\times10^6$ dark matter halos.

  The idea is to connect the observed GLF with the DMH catalog. We assume that  
each halo contains one and just one galaxy, and its luminosity is given by:
  \begin{equation}
  L_\textrm{galaxy}=\alpha M_\textrm{halo}^\beta 
  \end{equation}
but this function is given in magnitude units. Is necesary to convert 
magnitude to luminosity.

  \subsection{Magnitude to Luminosity}
  The Luminosity is an intrinsic propertie of the stars. It doesn't deppends of 
the distance.  It's directly related to the energy flux emmited. The luminosity 
$L_\nu$ at a given frecuency $\nu$ has $[\textrm{W}\textrm{Hz}^{-1}]$ or 
$[\textrm{erg}\textrm{s}^{-1}\textrm{Hz}^{-1}]$ units.

  The magnitude is brightnes star classification inherited from ancient Greeks. 
It deppends of the stellar distance. The absolute magnitude is a modern 
classification independent of  distance. Taking the Sun as reference, the 
absolute magnitud at a given wavelengt is given by: 
 \[ M_{\lambda} = M_{\lambda \odot} - 2.5 \log_{10}\left( 
\frac{L_\lambda}{L_{\lambda \odot}} \right) \]
  The solar absolute magnitude in the U filter is $M_{U\odot} = 5.61$,
and the solar luminosity in the same filter is $L_{U\odot} = 10^{18.48} 
\textrm{ergs s}^{-1}\textrm{Hz}^{-1}$ or $ L_{U\odot} = 3.02 \times 10^{18} 
\textrm{ergs s}^{-1}\textrm{Hz}^{-1}$.

  Replacing in the absolute magnitude equation:
  \[ M_{U} = 5.61 - 2.5 \log_{10}(L_{U}) + 2.5\times18.48\]
  gives the magnitude in the U filter of an astrophysical source:
  \[ M_{U} = 51.82 - 2.5 \log_{10}(L_{U}) \]

  \subsection{Best Fitting Parameters}



\section{The Fitting Model}

  This fitting model contains four parameters: $\left(m/M\right)_0$, $M_1$, 
$\beta$ and $\gamma$, 
  where $m:=$ Stellar mass, and $M :=$ Dark Matter Halo mass.
  \begin{equation}
  \frac{m}{M} = 2 \left( \frac{m}{M} \right)_{0} 
		    \left[ \left(\frac{M}{M_1}\right)^{-\beta} + 
\left(\frac{M}{M_1}\right)^{\gamma} \right]^{-1} 
  \end{equation}
  This is similar to the proposed by \cite{moster10}.
  Another model cited in the article contains five parameters: $m_0$, $M_1$, 
$\beta$, $\gamma_1$ and $\gamma_2$.
  \[ m(M) = m_0 \frac{ (M/M_1)^{\gamma_1}}{ \left[ 1 + (M/M_1)^\beta \right]^{ 
(\gamma_1-\gamma_2)/\beta}} \]


   ``We use a statistical approach to determine the relationship between the 
stellar masses of galaxies and the masses
  of the dark matter halos in which they reside. We obtain a parameterized 
stellar-to-halo mass (SHM) relation by
  populating halos and subhalos in an N-body simulation with galaxies and 
requiring that the observed stellar mass
  function be reproduced.'' \citep{moster10}

\subsection{Star Formation Rate}

The age of a star can be estimated by analyzing its spectrum. But when far
galaxies are studied, individual stars can not be resolved. Is not 
possible to make a detailed census of the galaxy popullation. Only is possible 
to get information from the whole stellar population, an integrated spectrum.

There is a  method\citep{madau98} in wich a linear relation between SFR
and luminosity in specific wavelength ranges can be assumed. This model allows
to estimate the young stars fraction and the mean SFR over periods of $10^8 -
10^9 \textrm{yr}$\citep{kennicutt98}. The luminosity in the model, comes from
the UV and the FIR broadband, also from speciffic recombination lines.

%``The UV continuum emission from a galaxy with significant ongoing star
%formation is entirely dominated by late-O/early-B stars on the main sequence.
%As these have masses $>10 M_\odot$ and lifetimes $t_{ms} 2\times 10^7 yr$, the
%measured luminosity becomes proportional to the stellar birthrate and
%independent of the galaxy history for $t\gg t_{ms}$ .'' MADAU 1998
%\citep{madau98}


In a typical galaxy spectrum the visible wavelengths are dominated by the main 
sequence stars (A to early F) and G-K giants. In few wavelength ranges we have 
a significative contribution from the young stars rather than the old stars. 
The infrared and far infrared wavelengths emission is dominated by dust, this 
dust is  heated by the whole stellar popullation, in particular by young, 
UV-bright stars \citep{law11}.

On active galaxies the UV broadband emission is dominated by late-O
and early-B type stars, with temperature near to 40.000K. These hot and massive
stars has a lifetime below $10^7\textrm{Gyr}$, they spend their nuclear fuel
faster than smaller and cooler sunlike stars.

The relation between UV luminosity and Star Formation Rate
\citep{madau98,kennicutt98} 
is given by:
\begin{equation}
 \textrm{SFR}\left(M_\odot \textrm{yr}^{-1}\right) 
      = 1.4 \times 10^{-28} L_{\nu} \left( \textrm{erg s}^{-1}\textrm{Hz}^{-1} 
	\right)
\end{equation}


  With Initial Mass Function (IMF) between $0.1 M_\odot$ 
 and $100 M_\odot$, in the range of $1250-2500 \mathring{\textrm{A}} $

The star forming rate will be:
  \begin{equation}
  SFR = k \times L_{0} M \left[ \left( \frac{M}{M_0}\right)^{-\beta} 
		   + \left( \frac{M}{M_0}\right)^{\gamma} 
               \right]^{-1}
  \end{equation}

The UV dust absorption \citep{kennicutt09} is not taken account in this work.
%DUST EXTINTION ALSO STUDIED by 002-TACCHELLA-A Physical Model for ---

%Synthesis Models:
%1) Study individual star spectra, star atmospheres. Creation of spectral 
%libraries.
%2) ``Individual stellar templates are summed together weigthed by an initial 
%mass function (IMF)... These isochrones can the be added in linear combination 
%to syntesize the spectrum or colors of a galaxy with any arbitrary star 
%formation history, usually parametrized as an expinential function of time.
%*** A single model contains at least 4 parameters (SF history, galaxy age, 
%metal abundance, and IMF). The colors of normal galaxies are well represented 
%by a one-parameter sequence with fixed age, composition and IMF, varying in the 
%time dependence of the SFR****

\section{Observations}

This paper is based on four main observational data sets. The most recent is
from the Hubble Space Telescope Legacy plus ground telescopes \citep{bouwens14}.
The second set is from the Canada-France Hawaii Telescope\citep{willott13} and
the last one by the UK Infrared Telescope and the Subaru Telescope
\cite{mclure09}. Those observations where made using the Drop-out technique.

The data from the HSTL is a compilation of previous works since
2006 \citep{bouwens06}, wich includes also observations after the 2009
upgrade mission\citep{bouwens12}. They estimate the uncertainty in their LF due
to cosmic variance. Taking account they used five independent surveys, they get
jus $\sigma\sim 10\%$\citep{bouwens14} using the Cosmic Variance Calculator
developed by Trenti \& Stiavelli (2008).



The dataset was retrieved from graphs for \cite{bouwens14} and \cite{mclure09} 
using
GAVO-DEXTER\footnote{\url{http://dc.zah.uni-heidelberg.de/dexter/ui/ui/custom}}.




Willott et al. \cite{willott13} presented the sixth release of the Canada-France-Hawaii Telescope Legacy Survey CFHTLS. The observations where performed over four separated fields covering a total area $\sim 4 \deg^2$ (a large area), it gives this survey great robustness.
Optical observations used MegaCam with $u^* g' r' i' z'$ filters. The main selection criteria: all the objects must be brighter than magnitude $z' = 25.3$. The final number of LBGs founded was 40. Moreover, they get spectroscopic confirmation for 7 candidates using GMOS spectrograph on the Gemini Telescopes, which has a field of view $\ll 1 \deg^2$. They show incompleteness in the sample due to foreground contamination and the detection algorithm; there is no warranty to have every object brighter than the limit magnitude on the faint limit. The full galaxy LF at $z=6$ cannot be obtained as in other studies. Nevertheless, this survey was focussed on the highly luminous LBGs. LF is calculated using the stepwise maximum likelihood method of Efstathiu et al. [Cite required!!!], within magnitudes from $M_{1350} = -22.5$ up to $-20.5$. The luminosity function of $z=5.9$ shows an exponential decline at the bright end, where feedback processes and inefficient last cooling limites star forming in bright galaxies hosted in the most massive halos.






\begin{figure}
\epsscale{1.00} 
\plotone{fig/observational_data.pdf}
\caption{Observational data from \cite{bouwens14,mclure09}and \cite{willott13}.}
\label{graph_observational_data}
\end{figure}

     
\subsection{The Drop-out Technique - Lyman Break Technique}

\cite{steidel03}

\section{Discussion}

%\citep{lundgren14} SFR evolution from $z=1$ to $6$

%\citep{bouwens14} HST Legacy

%\citep{jiang11} Keck pectroscopy


%``Fig. 16.— Updated Determinations of the derived SFR (left axis) and U V
%luminosity (right axis) densities versus redshift (5.4). The
%left axis gives the SFR densities we would infer from the measured luminosity
%densities, assuming the Madau et al. (1998) conversion
%factor relevant for star-forming galaxies with ages of 108 yr (see also
%Kennicutt 1998). The right axis gives the U V luminosities we infer
%integrating the present and published LFs to a faint-end limit of −17 mag (0.03
%L∗ )''BOUWENS 2014 UV LF

\section{Summary}


\acknowledgments
%Gracias Totales
%
\appendix

%\section{Appendix material}
%DEXTER \url{http://dc.zah.uni-heidelberg.de/dexter/ui/ui/custom}

\begin{thebibliography}{}
\bibitem[Bouwens(2006)]{bouwens06} Bouwens, R. J. et al. 2006, \apj, 653, 53	
		%2006ApJ...653...53B
\bibitem[Bouwens(2012)]{bouwens12} Bouwens, R. J. et al. 2012, \apj, 752, 5 	
		%2012ApJ...752L...5B
\bibitem[Bouwens(2014)]{bouwens14} Bouwens, R. J. et al. 2012, arXiv:1403.4295
\bibitem[Kennicutt(1998)]{kennicutt98} Kennicutt, Robert C., Jr. 1998, \araa, 
36, 189		%1998ARA&A..36..189K
\bibitem[Kennicutt(2009)]{kennicutt09} Kennicutt, Robert C., Jr et al. 2009, \apj, 
703, 4672	%2009ApJ...703.1672K
\bibitem[Law(2011)]{law11} Law, K. et al. 2011, \apj, 738, 124
		%2011ApJ...738..124L
\bibitem[Jiang(2011)]{jiang11} Jiang, Linhua et al. 2011, \apj, 743, 65
		%2011ApJ...743...65J
\bibitem[Lundgren(2914]{lundgren14} Lundgren, Britt F. et al, 2014, \apj, 780,34
		%2014ApJ...780...34L
\bibitem[McLure(2009)]{mclure09} McLure, R. J. et al. 2009, \mnras, 395, 2196	
		%2009MNRAS.395.2196M
\bibitem[Moster(2010)]{moster10} Moster, Benjamin P. et al. 2010, \apj, 710, 
903 
\bibitem[Riebe(2013)]{riebe13} Riebe, K. et al. 2013, AN, 334, 691 		
		%2013AN....334..691R
\bibitem[Steidel(2003)]{steidel03} Steidel, Charles C. et al. 2003, \apj, 592, 
728 		%2003ApJ...592..728S
\bibitem[Trimble(1987)]{trimble87} Trible, Virginia. 1987, \araa, 25, 425
\bibitem[van den Bosch(203)]{vandenbosch03} van den Bosh, Frank C. et al. 2003, \mnras,
40, 771		%2003MNRAS.340..771V
\bibitem[Willott(2013)]{willott13} Willott, Chris J. et al. 2013, \aj, 145, 4	
		%2013AJ....145....4W


\end{thebibliography}


\end{document}

%%
%% End of file `sample.tex'.
