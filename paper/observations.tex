\section{Observations}

This paper is based on three main observation sets. The most recent is from the
Hubble Space Telescope, this observations where performed by \cite{bouwens14}.
The second set is from the Canada-France Hawaii Telescope, \cite{willott13} and
the last one by the UK Infrared Telescope and the Subaru Telescope
\cite{mclure09}. Those observations where made using the Frop-out technique.

The data from the HST is a compilation of previous works since
2006 \citep{bouwens06}, wich includes also observations after the 2009
upgrade mission\citep{bouwens12}. 

The dataset was retrieved from graphs for \cite{bouwens14} and \cite{mclure09} 
using
GAVO-DEXTER\footnote{\url{http://dc.zah.uni-heidelberg.de/dexter/ui/ui/custom}}.



\begin{figure}
\epsscale{1.00} 
\plotone{fig/observational_data.pdf}
\caption{Observational data from \cite{bouwens14,mclure09}and \cite{willott13}.}
\label{graph_observational_data}
\end{figure}

     
\subsection{The Drop-out Technique - Lyman Break Technique}

\cite{steidel03}