\section{Observations}

This paper is based on four main observational data sets from the Hubble Space Telescope 
and three ground-based telescopes. All $z\sim 6$ LBG candidates where discovered using using 
the drop-out technique\citep{steidel03}. All magnitudes are in AB system.
%%%%%%%%%%%%%%Steidel 1996, Bunker 2003

The data from the Hubble Space Telescope Legacy (HSTL) \citep{bouwens14} is a compilation 
of observations since the installation of the Advanced Camera for Surveys (ACS) in 2002, 
through the near-infrared Wide Field Camera 3 (WFC3/IR) installed in 2009, up to 2012. 
The HST fields of view are: XDF, HUDF09-1, HUDF09-2, CANDELS-S/Deep, 
CANDELS-S/Wide, ERS, CANDELS-N/Deep, CANDELS-N/Wide, CANDELS-UDS, 
CANDELS-COSMOS and CANDELS-EGS, with areas of 4.7, 4.7, 4.7, 64.5, 34.2, 
40.5, 62.9, 60.9, 151.2, 151.9 and 150.7 arcmin$^2$ respectively. The total area corresponds 
to $\sim 0.7 \deg^2$ over five different lines of sight, reducing cosmic variance effects. 
Two cameras performed the observations: ACS and WFC3/IR, using 
$B_{435}$, $V_{606}$, $ i_{814}$, $ z_{850}$, $ I_{814}$, $Y_{098}$, $Y_{105}$, $J{125}$, 
$JH_{140}$ and $H_{160}$ filters.
The limit magnitude is between $\sim27.5$mag in CANDELS-EGS and$\sim30 $mag in the 
deepest field (XDF). Total number of z=6 LBG candidates is 940, most of them in the faint 
end of the LF, magnitudes in the rest frame are in the range $-22.52\leq M_{1600} \leq -16.77$
\citet{bowens14} calculated LF using a stepwise maximum-likelihood (SWML) based on \citet{efstathiou88}. 
The Schechter parameters derived are: 
$\phi^* =(0.33_{-0.10}^{+0.15}) \times 10 ^{-3} / \textrm{Mpc}^{-3} $, 
$M^*_{1600} = -21.16\pm 0.20$ and $\alpha = -1.91 \pm 0.09$. 
\citet{bouwens14} reported that using just few fields of view, UVLF has a slightly non-Schechter-like form. 



\citet{finkelstein14} worked also with HST, using the HUDF, CANDELS and GOODS fields, 
along with two of the Hubble Frontier Fields (HFF): deep parallel observations (unlensed 
fields) near the Abell 2744 and MACS J0416.1-2403 clusters. The HFF uses the ACS and 
the WFC3/IR with the same filters aforementioned but $z_{850}$. 



\citet{willott13} presented the sixth release of the Canada-France-Hawaii 
Telescope Legacy Survey CFHTLS. The observations where performed over four 
separated fields covering a total area $\sim 4 \deg^2$ (a large area), it gives this 
survey great robustness. Optical observations used MegaCam  %%%%%%[CITE REQUIRED] 
with $u^* g' r' i' z'$ filters. The main selection criteria: all the 
objects must be brighter than magnitude $z' = 25.3$. The final 
number of LBGs founded was 40. Moreover, they get spectroscopic confirmation 
for 7 candidates using GMOS spectrograph on the Gemini Telescopes, which 
has a $\ll 5.5$-square arcmin field of view . They show incompleteness in the sample due to 
foreground contamination and the detection algorithm; there is no warranty to 
have every object brighter than the limit magnitude on the faint limit. The full 
galaxy LF at $z=6$ cannot be obtained as in other studies. Nevertheless, this survey 
was focussed on the highly luminous LBGs. LF is calculated using the stepwise 
maximum likelihood method of \citet{efstathiou88}, within 
magnitudes 
from $M_{1350} = -22.5$ up to $-20.5$. The luminosity function of $z=5.9$ shows 
an exponential decline at the bright end, where feedback processes and inefficient 
last cooling limites star forming in bright galaxies hosted in the most massive halos.

\citet{mclure09} build the luminosity function for $z=5$ and $z=6$ using data from two ground-based
 telescopes: the United Kingdom Infrared Telescope in the near-IR imaging and Subaru 
 Telescope for the optical imaging. They use the first data release of the UKIRT Infrared 
 DeepSky Survey Ultra Deep Survey (UDS), together with the Subaru XMM-Newton 
 Survey (SXDS). Total observed area is $0.63 \deg^2$ uniformly covered by booth catalogues.
The UKIRT was equipped with the WFCAM using $J K$ filters. The Subaru was equipped 
with the Suprime-Cam with the $B V R i' z'$ filters. All candidates where brighter than 
$z'=26$. The UV rest frame magnitude range is $-22.4\leq M_{1500} \leq-20.6$. The LF 
was calculated using the maximum likelihood estimator of \citet{schmidt68}. 
Their analysis gave a total number of 104 LBG candidates in the redshift range 
$5.7\leq z \leq 6.3$. LF was parameterized according to the Schechter function with 
$\phi^* / \textrm{Mpc}^{-3} =(1.8 \pm 0.5) \times 10 ^{-3}$, $M^*_{1500} = -20.04\pm 0.12$ 
and $\alpha = -1.71 \pm 0.11$. 

The dataset was retrieved from \cite{mclure09} graph using
GAVO-DEXTER\footnote{\url{http://dc.zah.uni-heidelberg.de/dexter/ui/ui/custom}}.




\begin{figure}
\epsscale{1.00} 
\plotone{fig/observational_data.pdf}
\caption{Observational data from \cite{bouwens14,mclure09}and \cite{willott13}.}
\label{graph_observational_data}
\end{figure}

%They estimate the uncertainty in their LF due
%to cosmic variance. Taking account they used five independent surveys, they get
%#jus $\sigma\sim 10\%$\citep{bouwens14} using the Cosmic Variance Calculator
%developed by Trenti \& Stiavelli (2008).
