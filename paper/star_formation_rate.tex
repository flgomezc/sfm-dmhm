\subsection{Star Formation Rate}

The age of a star can be estimated by analyzing its spectrum. But when far
galaxies are studied, individual stars can not be resolved. Is not 
possible to make a detailed census of the galaxy popullation. Only is possible 
to get information from the whole stellar population, an integrated spectrum.

There is a  method\citep{madau98} in wich a linear relation between SFR
and luminosity in specific wavelength ranges can be assumed. This model allows
to estimate the young stars fraction and the mean SFR over periods of $10^8 -
10^9 \textrm{yr}$\citep{kennicutt98}. The luminosity in the model, comes from
the UV and the FIR broadband, also from speciffic recombination lines.

%``The UV continuum emission from a galaxy with significant ongoing star
%formation is entirely dominated by late-O/early-B stars on the main sequence.
%As these have masses $>10 M_\odot$ and lifetimes $t_{ms} 2\times 10^7 yr$, the
%measured luminosity becomes proportional to the stellar birthrate and
%independent of the galaxy history for $t\gg t_{ms}$ .'' MADAU 1998
%\citep{madau98}


In a typical galaxy spectrum the visible wavelengths are dominated by the main 
sequence stars (A to early F) and G-K giants. In few wavelength ranges we have 
a significative contribution from the young stars rather than the old stars. 
The infrared and far infrared wavelengths emission is dominated by dust, this 
dust is  heated by the whole stellar popullation, in particular by young, 
UV-bright stars \citep{law11}.

On active galaxies the UV broadband emission is dominated by late-O
and early-B type stars, with temperature near to 40.000K. These hot and massive
stars has a lifetime below $10^7\textrm{Gyr}$, they spend their nuclear fuel
faster than smaller and cooler sunlike stars.

The relation between UV luminosity and Star Formation Rate
\citep{madau98,kennicutt98} 
is given by:
\begin{equation}
 \textrm{SFR}\left(M_\odot \textrm{yr}^{-1}\right) 
      = 1.4 \times 10^{-28} L_{\nu} \left( \textrm{erg s}^{-1}\textrm{Hz}^{-1} 
	\right)
\end{equation}


  With Initial Mass Function (IMF) between $0.1 M_\odot$ 
 and $100 M_\odot$, in the range of $1250-2500 \mathring{\textrm{A}} $

The star forming rate will be:
  \begin{equation}
  SFR = k \times L_{0} M \left[ \left( \frac{M}{M_0}\right)^{-\beta} 
		   + \left( \frac{M}{M_0}\right)^{\gamma} 
               \right]^{-1}
  \end{equation}

\subsection{Dust Absorption}
%The UV dust absorption \citep{kennicutt09} is not taken account in this work.
%DUST EXTINTION ALSO STUDIED by 002-TACCHELLA-A Physical Model for ---

%Synthesis Models:
%1) Study individual star spectra, star atmospheres. Creation of spectral 
%libraries.
%2) ``Individual stellar templates are summed together weigthed by an initial 
%mass function (IMF)... These isochrones can the be added in linear combination 
%to syntesize the spectrum or colors of a galaxy with any arbitrary star 
%formation history, usually parametrized as an expinential function of time.
%*** A single model contains at least 4 parameters (SF history, galaxy age, 
%metal abundance, and IMF). The colors of normal galaxies are well represented 
%by a one-parameter sequence with fixed age, composition and IMF, varying in the 
%time dependence of the SFR****