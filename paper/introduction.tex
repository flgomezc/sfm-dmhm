\section{Introduction}

Dark matter is a significative component of the universe.

\citep{trimble87}

Hierarchy Structure Evolution: Early formation of small structures
merging on major structures after.

From simulations, Halo Mass Function as function of redshift (or time).

Star formation rate as function of time. Peak at $z\sim 2$.

Main Objetive: Reproduce the observed luminosity function at redshift $z=5.9$
from a DMH catalog from simulations. 

  \subsection{Halo Mass Function (HMF)}
  
%  The HMF is created using the DMH catalog from the Multidark Database 
%\citep{riebe13}.
  
  
  \subsection{Cosmic Variance}
  
%  Is important to know and understand the existence of local fluctuations   or 
%inhomogeneities due the observed scale.  For this part is necesary to study the 
%HMF on smaller boxes

\section{Linking Galaxy Luminosity Function (GLF) and
  Dark Matter Halo (DMH) mass}

%  In one hand we have observational results: the GLF for star-forming galaxies 
%at high redshift (z=5.9)\citep{bouwens06,willott13}  expressed in terms of 
%magnitude in the ultraviolet range (M1350). In the other hand we have a DMH  
%atalog, result from the Multidark Simulation with Plank Cosmology. 
%\citep{Multidark}

 % A $1000 \textrm{ Mpc} ^3 \textrm{h}^{−3}$ cubic box containing near to  
%$11\times10^6$ dark matter halos.

%  The idea is to connect the observed GLF with the DMH catalog. We assume that  
%each halo contains one and just one galaxy, and its luminosity is given by:
%  \begin{equation}
%  L_\textrm{galaxy}=\alpha M_\textrm{halo}^\beta 
%  \end{equation}
%but this function is given in magnitude units. Is necesary to convert 
%magnitude to luminosity.

\subsection{Magnitude to Luminosity}

Luminosity Functions (LF) are usually expressed in terms of magnitude instead 
luminosity. Luminosity is the energy emmited by a source in a given wavelengt range, 
is a physical quantity. 
%The luminosity 
%$L_\nu$ at a given frecuency $\nu$ has $[\textrm{W}\textrm{Hz}^{-1}]$ or 
%$[\textrm{erg}\textrm{s}^{-1}\textrm{Hz}^{-1}]$ units.
Magnitude is a classification inherited from ancient Greeks, this
quantifies the response of the first astrometric device: the human eye, this
perception grows loarithmically with the retrieved radiation.

Luminosity of any object can be compared with Sun Luminosity $(L_{\lambda
\odot})$ at any wavelengt. With the Sun Magnitude as reference
$(M_{\lambda \odot})$, the absolute magnitud of the objet  at an specific
wavelengt is given by: 
 \[ M_{\lambda} = M_{\lambda \odot} - 2.5 \log_{10}\left( 
\frac{L_\lambda}{L_{\lambda \odot}} \right) \]
  The solar absolute magnitude in the U filter is $M_{U\odot} = 5.61$,
and the solar luminosity in the same filter is $L_{U\odot} = 10^{18.48} 
\textrm{ergs s}^{-1}\textrm{Hz}^{-1}$ or $ L_{U\odot} = 3.02 \times 10^{18} 
\textrm{ergs s}^{-1}\textrm{Hz}^{-1}$. The solar luminosity can be used as
refference unit, in this fashion, the typical luminosity of a galaxy can be expresed in
terms of $10^{8}-10^{11}$ times the sun luminosity.

  The absolute magnitude of a galaxy equation in the U filter:
  %\[ M_{U} = 5.61 - 2.5 \log_{10}(L_{U}) + 2.5\times18.48\]
  %gives the magnitude in the U filter of an astrophysical source:
  \[ M_{U} = 51.82 - 2.5 \log_{10}(L_{U}) \]

 %\subsection{Best Fitting Parameters}