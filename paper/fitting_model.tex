\section{The Fitting Model}

  This fitting model contains four parameters: $\left(m/M\right)_0$, $M_1$, 
$\beta$ and $\gamma$, 
  where $m:=$ Stellar mass, and $M :=$ Dark Matter Halo mass.
  \begin{equation}
  \frac{m}{M} = 2 \left( \frac{m}{M} \right)_{0} 
		    \left[ \left(\frac{M}{M_1}\right)^{-\beta} + 
\left(\frac{M}{M_1}\right)^{\gamma} \right]^{-1} 
  \end{equation}
  This is similar to the proposed by \cite{moster10}.
  Another model cited in the article contains five parameters: $m_0$, $M_1$, 
$\beta$, $\gamma_1$ and $\gamma_2$.
  \[ m(M) = m_0 \frac{ (M/M_1)^{\gamma_1}}{ \left[ 1 + (M/M_1)^\beta \right]^{ 
(\gamma_1-\gamma_2)/\beta}} \]


   ``We use a statistical approach to determine the relationship between the 
stellar masses of galaxies and the masses
  of the dark matter halos in which they reside. We obtain a parameterized 
stellar-to-halo mass (SHM) relation by
  populating halos and subhalos in an N-body simulation with galaxies and 
requiring that the observed stellar mass
  function be reproduced.'' \citep{moster10}