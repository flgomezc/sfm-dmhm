\subsection{The Luminosity Model}

In this model we have made two assumptions:
\begin{enumerate}
 \item Each halo in the catalog hosts one galaxy. There are not empty
halos, also none of halos has two or more galaxies.
 \item The UV luminosity of each galaxy is function of one variable: the mass of
the DMH in wich is located.
\end{enumerate}

The simplest relation we can have is a powerlaw:
 \begin{equation}
  L = L_0 M^\alpha
 \end{equation}
but has not well agreement with observed luminosity functions.

A better model is a four parameter function. Each galaxy has a luminosity given
by:
  \begin{equation}
  L = L_{0} M \left[ \left( \frac{M}{M_0}\right)^{-\beta} 
		   + \left( \frac{M}{M_0}\right)^{\gamma} 
               \right]^{-1}
  \end{equation}
where $M$ is the hosting DMH mass, $L_{0}$ is a normalization constant, $M_0$
is the critical mass where the luminosity function has a slope change, 
$\beta$ and $\gamma$ are the slopes. This equation has a similar fashion to the
mass to light relation \citep{vandenbosch03} and the mean relation between
stellar mas of a galaxy and the mas of its halo used by \cite{moster10}.

There are more complex models\citep{lee09} that includes a random behavior:
galaxies has not synchronization on the beginning of star forming stage, also
this stage may be time limited. This is called duty cycle. It is probable to
have in the observations some invisible galaxies in the UV continuum due their
duty cycle may has not started as well it may ended. Also may be present a
normal distribution of the luminosity around the expected values.